
\chapter{Introduction}

Introduction chapter. We recommend using bibtex to manage your list
of references and organize the literature citation style. All is done
automatically by \LyX{}. While working on separate files it's useful
to include the bibtex file in each file locally, in the yellow note
(so it's recognized by the compiler but not complaining if you run
the Master document). 


\section{Figures}

Include figures as usual, first Insert -> Float -> Figure, then Insert
-> Graphics. 

Use cross-reference option to talk about Fig. \ref{fig:Some-caption-for}

\begin{figure}
\begin{centering}
\includegraphics[width=0.8\textwidth]{few_long_trajectories}
\par\end{centering}

\caption{Some caption for the figure. Don't forget adding label to the figure
and later using the cross-reference to it. \label{fig:Some-caption-for}}


\end{figure}



\section{Tables}

\begin{table}
\begin{tabular}{|c|c|c|}
\hline 
row 1 & cell 1 & cell 2\tabularnewline
\hline 
\hline 
row 2 & cell 1 & cell 2\tabularnewline
\hline 
\end{tabular}

\caption{Table float first using Insert - > Float -> Table}


\end{table}


\begin{comment}
Use: Insert Note -> Insert List/TOC -> Bibtex Bibliography
\end{comment}




\begin{comment}
Then in the text you can use Insert citation (use the icon for the
quick use)
\end{comment}

